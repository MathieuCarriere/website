\documentclass[a4paper, 11pt]{article}
 
\usepackage[utf8]{inputenc}    
\usepackage[T1]{fontenc}
\usepackage[english]{babel}
\usepackage{amsmath}
\usepackage{amssymb}
\usepackage{mathrsfs}
\usepackage{graphicx}
\usepackage{geometry}
\geometry{top=2 cm, bottom=2 cm, left=2.5 cm, right=2.5 cm}

\begin{document}
 
\title{Rapport de l'Ecole Mathematical Methods for High Dimensional Data Analysis}
\author{Mathieu Carri\`ere}
\date{}

\maketitle

Cette école d'été a pris place à Munich. Elle était destinée aux étudiants en science 
des données, avec un focus particulier sur les méthodes pour les données de grande dimension.
Cette école s'inscrit donc très naturellement dans le cadre de mes recherches, car ma thèse porte
aussi sur la science des données. L'aspect grande dimension n'est pas tout à fait le centre de 
mes recherches car je me concentre plutôt sur les données complexes, mais les méthodes présentées
pendant l'école présentent bien évidemment un grand intérêt pour mes travaux.

\begin{center} \textbf{Cours 1. Streaming and Sketching Algorithm (J. Nelson)} \end{center}

Un sketch de données, étant donnée une famille de queries, est une compression de
ces données permettant tout de même de donner une réponse définitive à ces queries.
Ces sketches peuvent être conservés lors de l'ajout (ou suppression) des données initiales :
c'est le paradigme du streaming. Le cours présentait quelques exemples d'algorithmes de
sketching et de streaming, avec une point de vue très théorique. L'illustration des avantages
à utiliser de tels algorithmes (réduction espace et mémoire) a été démontrée formellement
au tableau mais aussi durant une séance d'exercices à la fin du cours.

\begin{center} \textbf{Cours 2. Topological Time Series Analysis (J. Perea)} \end{center}

Le cours présentait une analyse précise des séries temporelles en transformant la série en un nuage
de points en grande dimension. Plus spécifiquement, le théorême de plongement de Takens précise
qu'il est possible de construire une isométrie entre les points d'une variété lisse et des points
dans un espace euclidien de grande dimension. C'est précisément cette propriété qui est utilisée, 
et la géométrie du nuage de points obtenus permet d'inférer des propriétés de la série temporelle.
Par exemple, quand la série est constituée de la somme de deux sinus, le rapport entre les périodes
influence la topologie du nuage : si ce rapport est rationnel, le nuage a la topologie d'un cercle,
tandis que si ce rapport est irrationnel, le nuage a celle d'un tore. 
La topologie d'un nuage de points se calculant aisément via la théorie de l'homologie, 
de nombreuses applications ont été
présentées, notamment en analyse de séries d'images (vidéo) et de série musicale.

\begin{center} \textbf{Cours 3. Optimal Stochastic Regularization for Large Scale Machine Learning (L. Rosasco)} \end{center}

Ce cours présentait une approche unifiée et systématique pour l'étude l'apprentissage automatique. Toutes les techniques
et les algorithmes habituels ont en effet été réintroduits comme des étant des solutions de problèmes de minimisation
pour différentes fonctionnelles de perte. Les machines à support de vecteurs (SVM) par exemple, peuvent se comprendre
comme les solutions analytiques qui minimisent la fonctionnelle hinge loss. Plus particulièrement, les régularisations
des énergies à minimiser peuvent se comprendre comme une minimisation de la complexité du classificateur solution
permettant d'éviter le surapprentissage sur les données. De nombreuses techniques d'optimisation et d'exemples
de fonctionnelles ont été présentées et testées sur des jeux de données pour comprendre l'influence des paramètres 
lorsque les données sont en dimension élevée. 

\end{document}