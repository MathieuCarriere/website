\documentclass[a4paper, 11pt]{article}
 
\usepackage[utf8]{inputenc}    
\usepackage[T1]{fontenc}
\usepackage[francais]{babel}
\usepackage{amsmath}
\usepackage{amssymb}
\usepackage{mathrsfs}
\usepackage{graphicx}
\usepackage{geometry}
\geometry{top=2 cm, bottom=2 cm, left=2.5 cm, right=2.5 cm}

 
\begin{document}
 
\title{Rapport des Journées de Géométrie Algorithmique 2015}
\author{Mathieu Carrière}
\date{}
 
\maketitle

Cette \'Ecole a eu lieu au milieu de mois de Novembre et a réuni beaucoup de chercheurs et doctorants qui travaillent en g\'eom\'etrie combinatoire, algébrique et différentielle mais aussi en analyse (équations aux dérivées partielles). De manière générale, les cours proposés avaient tous un lien plus ou moins
direct avec la géométrie
qui est le domaine mathématique auquel je suis le plus souvent confronté dans mon travail de thèse.  

\begin{center} \textbf{Cours 1. Applications et enjeux en géométrie algorithmique issus de la discrétisation des équations aux dérivées partielles (J.M. Mirebeau)} \end{center}

Le cours est en 3 parties, dont le dénominateur commun concerne
la discrétisation des équations aux dérivées partielles (EDP).

La première partie se penche sur les
discrétisations adaptatives des EDP et le formalisme maillage métrique.
La problématique globale est celle de l'anisotropie des 
solutions des EDP. En effet, de telles solutions sont souvent de régularité non-uniforme: la fonction solution peut être constante sur des régions entières du domaine, et varier de manière complexe et significative sur des zones singulières (points, surfaces).
Pour pallier à ce phénomène,
le maillage du domaine doit être fortement non-uniforme, et satisfaire des contraintes locales de taille et de formes.
Des résultats sur l'estimation, en fonction de la taille du maillage, de l'erreur faite en interpolant la solution par des fonctions linéaires par morceaux sur les mailles ont été présentés.

La deuxième partie du cours était axée sur l'analyse de formes, qui se base très fréquemment sur les EDP, pour le débruitage, la reconstruction, ou encore la segmentation en régions. Les opérateurs de ces EDP encodent souvent la géométrie et l'anisotropie de ces formes. En revanche, leur discrétisation requiert des outils de géométrie des réseaux. La structure additive des grilles de pixels/voxels représentant les formes a été présentée et utilisée en ce sens.

La dernière partie se concentrait sur
les diagrammes de Voronoi et leur généralisation, les diagrammes de puissance, qui sont des outils fondamentaux en discrétisation des EDP.
Quelques applications ont été présentées, telles que le transport optimal, la mécanique des fluides et des modèles économiques.



\begin{center} \textbf{Cours 2. Tangent Vector Fields (M. Desbrun)} \end{center}

Ce cours était très intéressant et instructif car il expliquait clairement une problématique fondamentale en Computer Graphics : la discrétisation et la représentation associée des champs de vecteurs tangents. En effet, si la notion de plan tangent est bien connue en géométrie différentielle, la question de sa définition est bien moins établie quand le domaine est discrétisé (échantillonnage d'une variété par exemple). De plus, les plans tangents permettent à leur tour de définir une multitude d'objets très importants, comme les connections, les courbes intégrales ou le gradient. Trois options ont été introduites pour le cas des maillages durant le cours. La première est basée sur les faces et définit le plan tangent à une face comme le plan formé par le triangle correspondant. La deuxième est une version arète, dans laquelle un plan tangent est interpolé avec des fonctions de Whitney linéaires par morceaux. Enfin, le troisième méthode est une version point. Pour cette dernière version, on se sert du voisinage direct du point dans le 1-squelette pour approcher une carte locale de la variété et déduire un plan tangent.

\newpage

\begin{center} \textbf{Cours 3. Géométrie riemannienne des matrices symétriques définies positives et applications (M. Moakher)} \end{center}

Les matrices symétriques définies positives jouent un rôle fondamental dans plusieurs applications des mathématiques, comme l'imagerie médicale. Le cours se voulait un résumé des outils de manipulation et de traitement spécifiques pour ce type de matrices. Il commence par détailler la structure de l'ensemble de ces matrices, qui n’est pas un espace vectoriel mais une variété différentielle, sur laquelle peuvent s'applique les méthodes de la géométrie riemannienne.
En particulier, des expressions explicites du tenseur métrique, de la dérivée covariante, des symboles de Christoffel, de la courbure, et de divers opérateurs différentiels ont été présentés. De ces opérateurs découlent plusieurs notions de moyennes pour les matrices symétriques définies positives qui sont basées sur les différentes métriques.
Ces moyennes sont utiles pour présenter les différentes applications de la géométrie riemannienne des matrices symétriques définies positives en théorie d’élasticité et en IRM avec le tenseur de diffusion.
Ce cours traitait de sujets très similaires avec le cours de H. Delingette que j'ai suivi durant mon M2, même si le traitement mathématique des matrices SPD était moins poussé.

 

\begin{center} \textbf{Cours 4. Des molécules qui calculent (N. Schabanel)} \end{center}

Ce cours était une introduction à de nouveaux modèles de calcul qui sont implémentés en laboratoire et fonctionnent par le biais des molécules. Ces calculs sont obtenus en laissant des molécules à base d’ADN se replier sous forme de tuiles de taille nanoscopique, puis s’assembler par affinités pour réaliser des formes géométriques arbitraires de quelques centaines de nanomètres de diamètre. Le premier modèle effectif présenté est celui d’auto-assemblage, dont on peut démontrer qu’il peut implémenter n’importe quel calcul Turing. Différentes variantes ont ensuite été présentées aussi bien du côté des modèles théoriques, que des implémentations expérimentales. Ceux-ci ont permis la réalisation de smileys, cartes, alphabets, compteur binaire nanoscopiques, ainsi que les prémices d’une robotique nanoscopique à base d’ADN, donc potentiellement compatible avec la vie. Ces modèles reposent sur un mélange, entre autres, de géométrie discrète, de calculabilité, d’automates cellulaires et d’algorithmique classique et aléatoire, et permettraient d'avoir une puissance de calcul significative via des parallélisations massives.

\begin{center} \textbf{Cours 5. Topological tools for stable sets and coloring of graphs (S. Thomassé)} \end{center}

Ce cours a expliqué en détail la question du calcul du nombre chromatiques des graphes. Bien que très simple à définir (plus grand nombre de couleurs possibles pour colorier un graphe), obtenir des bornes inférieures sur sa valeur est extrêmement difficile. Le cours proposait des méthodes topologiques pour obtenir de tels résultats: la première méthode utilise le théorême de Borsuk-Ulam, la deuxième utilise la dimension de Vapnik-Cervonenkis et la dernière utilise la connectivité du complexe formé par les ensembles stables du graphe. Une partie conséquente du cours était de montrer en quoi un avancement non trivial sur ces questions permettrait de résoudre d'autres questions, liées de manière indirecte au problème, et dont la formulation est très simple, comme par exemple celle du nombre de cycles formés par des triangles dans les graphes orientés.

\end{document}
