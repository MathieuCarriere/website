\documentclass[a4paper, 11pt]{article}
 
\usepackage[utf8]{inputenc}    
\usepackage[T1]{fontenc}
\usepackage[english]{babel}
\usepackage{amsmath}
\usepackage{amssymb}
\usepackage{mathrsfs}
\usepackage{graphicx}
\usepackage{geometry}
\geometry{top=2 cm, bottom=2 cm, left=2.5 cm, right=2.5 cm}

 
\begin{document}
 
\title{Rapport de la Graduate School GeoStoch 2016}
\author{Mathieu Carri\`ere}
\date{}
 
\maketitle

Cette Ecole a eu lieu juste avant la conf\'erence de GeoStoch 2016 sur la g\'eom\'etrie stochastique, c'est-\`a-dire l'\'etude g\'eom\'etrique
d'objets construits sur des r\'ealisations de variables al\'eatoires, comme par exemple l'\'etude de l'homologie d'un complexe
simplicial construit \`a partir d'un nuage de points r\'ealis\'es suivant une certaine loi de probabilit\'e.
C'est un th\`eme qui est naturellement li\'e \`a mon travail de th\`ese, puisque je m'int\'eresse moi aussi \`a l'\'etude statistique et
g\'eom\'etrique d'objets al\'eatoires, notamment ceux qui proviennent de mesures $(a,b)$-standardes.

\begin{center} \textbf{Cours 1. Random Fields and Scale Invariance (H. Bierm\'e)} \end{center}

Ce cours portait sur les champs al\'eatoires, c'est-\`a-dire les processus al\'eatoires dont l'indice est \`a valeurs
dans $\mathbb{R}^d$, ainsi que la notion d'auto-similarit\'e. 
Dans une premi\`ere partie, plusieurs m\'ethodes pour d\'efinir des champs al\'eatoires \`a partir de processus
al\'eatoires r\'eels ont \'et\'e expos\'ees. Par exemple, l'int\'egrale de la covariance des projet\'es des indices sur la sph\`ere de
dimension inf\'erieure d\'efinit les processus de Levy-Chentsov. Le produit des covariances pour chaque coordonn\'ee des indices
est aussi une mani\`ere raisonnable de proc\'eder.
Dans une deuxi\`eme partie, la notion de stationnarit\'e a \'et\'e \'etablie, c'est-\`a-dire la p\'eriodicit\'e des processus. 
En particulier, les processus stationnaires du second ordre ont des fonctions de covariance dites de type positif,
qui sont compl\`etement caract\'eris\'ees par leur mesure spectrale via le th\'eorême de Bochner.
Dans une troisi\`eme partie, l'auto similarit\'e proprement dite a \'et\'e pr\'esent\'ee. Il s'agit d'une notion similaire
\`a celle des fractales, et dont les champs correspondants poss\`edent des fonctions de covariance particuli\`eres.
En particulier, les processus de Levy-Chantsov et les champs Browniens sont auto-similaires.
La dimension de Hausdorff est une mesure naturelle du coefficient d'auto-similarit\'e, et, dans une quatri\`eme partie,
les processus de Hölder ont \'et\'e introduits, via un r\'esultat d\'emontrant que la dimension de Hausdorff du graphe
de tels processus peut être convenablement estim\'ee.

\begin{center} \textbf{Cours 2. Finite Volume Gibbs Point Processes (D. Dereudre)} \end{center}

Ce cours portait sur les processus ponctuels, c'est-\`a-dire les processus \`a valeurs dans les sous-ensembles localement finis
de $\mathbb{R}^d$, et dont la cardinalit\'e de l'intersection avec un autre sous-ensemble suit une loi de Poisson.
La premi\`ere partie du cours se focalisait sur la d\'efinition de certains de ces processus, 
les processus de Gibbs, qui se d\'efinissent par une loi de forme tr\`es particuli\`ere qui fait intervenir
explicitement les lois de Poissons ainsi qu'un terme d'\'energie, dont la forme influence les r\'esultats (mod\`eles de Strauss, Lennard-Jones).
Cette loi est diff\'erentiable, et sa d\'eriv\'ee correspond en fait \`a la variance de ce même processus.
La deuxi\`eme partie du cours se concentrait la preuve de deux \'equations importantes.
L'\'equation DLR permet d'estimer la cardinalit\'e d'un volume fini, conditionnellement \`a celle du compl\'ementaire de ce volume.
L'\'equation GNZ, quant \`a elle, permet de calculer l'esp\'erance du processus sur ce même volume.
La derni\`ere partie du cours \'etait concentr\'ee sur les r\'esultats de convergence des processus de Gibbs
pour des topologies particuli\`eres. En particulier, une convergence locale peut être \'etablie.
La validit\'e de l'\'equation DLR pour le processus limite a \'et\'e d\'emontr\'ee.

\end{document}
