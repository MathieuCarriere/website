\documentclass[11pt,a4paper]{moderncv}

	\usepackage[OT1]{fontenc}
	\moderncvtheme[blue]{classic}
	\usepackage[top=1.1cm, bottom=1.1cm, left=0.5cm, right=1cm]{geometry}
	\usepackage{eurosym}
	\setlength{\hintscolumnwidth}{2.5cm}
	\firstname{Mathieu}
	\familyname{Carri\`ere}
	\title{Topological Data Analysis and Machine Learning}
	\email{mathieu.carriere3@gmail.com}
	\mobile{+1 917-941-5182}
	\extrainfo{Adress: 390 Ft Washington Ave, \\ 10033 NYC, USA \\ https://mathieucarriere.github.io/website \\ Skype: mathieu.carriere \\ French and American citizenship}
	
	\newlength\listtripleitemmaincolumnwidth

\makeatletter
\renewcommand*{\recomputecvlengths}{%
  \setlength{\quotewidth}{0.65\textwidth}%
  \setlength{\maincolumnwidth}{\textwidth-\separatorcolumnwidth-\hintscolumnwidth}%
  \setlength{\listitemmaincolumnwidth}{\maincolumnwidth-\listitemsymbolwidth}%
  \setlength{\doubleitemmaincolumnwidth}{\maincolumnwidth-\hintscolumnwidth-\separatorcolumnwidth-\separatorcolumnwidth}%
  \setlength{\doubleitemmaincolumnwidth}{0.5\doubleitemmaincolumnwidth}%
  \setlength{\listdoubleitemmaincolumnwidth}{\maincolumnwidth-\listitemsymbolwidth-\separatorcolumnwidth-\listitemsymbolwidth}%
  \setlength{\listdoubleitemmaincolumnwidth}{0.5\listdoubleitemmaincolumnwidth}%
  \setlength\listtripleitemmaincolumnwidth{.66\listdoubleitemmaincolumnwidth}%
  \renewcommand{\headwidth}{\textwidth}%
  \setlength{\parskip}{0\p@}%
}
\makeatother

\newcommand{\cvdoublecolumn}[2]{%
  \cvline{}{%
  \begin{minipage}[t]{\listdoubleitemmaincolumnwidth}#1\end{minipage}%
  \hfill%
  \begin{minipage}[t]{\listdoubleitemmaincolumnwidth}#2\end{minipage}%
 }%
}

\newcommand{\cvtriplecolumn}[3]{%
  \cvline{}{%
  \begin{minipage}[t]{\listtripleitemmaincolumnwidth}#1\end{minipage}%
  \hfill%
  \begin{minipage}[t]{\listtripleitemmaincolumnwidth}#2\end{minipage}%
  \hfill%
  \begin{minipage}[t]{\listtripleitemmaincolumnwidth}#3\end{minipage}%
 }%
}

\newcommand{\cvreference}[7]{%
  \textbf{#1}\newline% Name
  \ifthenelse{\equal{#2}{}}{}{#2\newline}%
  \ifthenelse{\equal{#3}{}}{}{#3\newline}%
  \ifthenelse{\equal{#4}{}}{}{#4\newline}%
  \ifthenelse{\equal{#5}{}}{}{#5\newline}%
  \ifthenelse{\equal{#6}{}}{}{\texttt{#6}\newline}%
  \ifthenelse{\equal{#7}{}}{}{#7}}

\begin{document}

	\maketitle

	\section{Education}
		
		\cventry{Now}{Postdoctoral Research Fellow}{Rabad\'an Lab, Columbia University, New York, USA}{}{}{}
		%\cventry{11/2017-08/2018}{Postdoctoral Research Scientist}{DataShape, Inria Saclay, Palaiseau, France}{}{}{}
		\cventry{08/2018}{Ph.D. in Applied Mathematics and Informatics}{DataShape, Inria Saclay, Palaiseau, France}{}{}
                {\textbf{Title:} On metric and statistical properties of topological descriptors for geometric data.} %\textbf{Advisor:} Steve Oudot.}		

		\cventry{02/2015}{Engineering Degree}{Ecole Centrale Paris, Ch\^atenay-Malabry, France}{}{}{}
                %{\textbf{Relevant modules:} Geometric and Topological Modeling, C++ Programming, Statistics and Data Mining, Convex Optimization.}

 		\cventry{12/2014}{M.Sc. in Mathematics, Vision and Learning}{ENS Cachan, Cachan, France}{}{}{} 
                %\textbf{Relevant modules:} Probabilistic Graphical Models, Geometry and Shape Space, Computer Vision and Learning 
                %(SVM, Boosting, Random Forests, Neural Networks).}
		

	\section{Research contributions and impact}

		\cventry{}{\normalfont My research focuses on Topological Data Analysis and statistical Machine Learning, with an application to biology and genomics.
		I contributed to the definition and analysis of topological descriptors, as well as on the use of 
                kernels and deep learning methods for them. 
                %I also contributed to the study of rates of convergence and confidence intervals for these representations. 
		This work resulted in several scientific articles in top conference proceedings
		and scientific journals, and has been used in different fields, like bioinformatics and computer graphics.
		I implemented my work in the Python/C++ GUDHI library and in various Python packages. 
                I have also contributed to the community as a reviewer for several conferences and journals 
                (ICML, JMLR, SoCG, SODA, JACT, DCG, JoCG, TKDE, GD), and through the organization of the New-York Applied 
		Topology Meeting Group at Columbia University}{}{}{}{} %I am also interested in applying deep learning to TDA.

	\section{Research Articles}

	\subsection{Machine Learning and Statistics with Persistence Diagrams}

		%\cventry{}{\normalfont Local Signatures using Persistence Diagrams. M. Carri\`ere, S. Oudot, M. Ovsjanikov}{\textit{HAL}, 2015}{}{}{}
		\cventry{-}{\normalfont\small Stable Topological Signatures for Points on 3D Shapes. M. Carri\`ere, S. Oudot, M. Ovsjanikov}
			  {\small\textit{SGP}, \normalfont{2015}}
			  %{\small \textit{Proceedings of the Eurographics Symposium on Geometry Processing}, 2015}
                          {}{}{\footnotesize Used topological descriptors to improve accuracy in point classification on 3D shape datasets.}

		%\cventry{}{\normalfont Structure and Stability of the 1-Dimensional Mapper. M. Carri\`ere, S. Oudot}{\textit{Proceedings of the 32nd Symposium on Computational Geometry}, 2016}{}{}{}
		\cventry{-}{\normalfont\small Sliced Wasserstein Kernel for Persistence Diagrams. M. Carri\`ere, M. Cuturi, S. Oudot}
                          {\small\textit{ICML}, \normalfont{2017}}
			  %{\small\textit{Proceedings of the International Conference on Machine Learning}, 2017}
			  {}{}{\footnotesize Derived a new kernel for Persistence Diagrams improving accuracy on various datasets, such as texture images. }

		%\cventry{-}{\normalfont\small Are Cycles Important Topological Features? A Case Study in Brain Image Classification. M. Carri\`ere, 
                %           C. Chen, F. Liu, X. Ni, T. Wu, J. Fan, R. Rabadan}
                %          {\small\textit{Preprint}, \normalfont{2019}}
                %          {}{}{\footnotesize Used Persistence Diagrams and Cycles to improve autism classification from MRIs.}

		\cventry{-}{\normalfont\small On the Metric Distortion of Embedding Persistence Diagrams into separable Hilbert spaces. M. Carri\`ere, U. Bauer}
                          {\small\textit{SoCG}, \normalfont{2019}}
                          {}{}{\footnotesize Established negative results about the equivalence of Persistence Diagram distances and Hilbert space metrics.}	

		\cventry{-}{\normalfont\small PersLay: A Neural Network Layer for Persistence Diagrams and New Graph Topological Signatures. M. Carri\`ere, 
                           F. Chazal, Y. Ike, T. Lacombe, M. Royer, Y. Umeda}
                          {\small\textit{AISTATS}, \normalfont{2020}}
                          {}{}{\footnotesize Defined general Neural Net architecture for Persistence Diagrams with application in graph classification.}

		\cventry{-}{\normalfont\small Persistent homology based characterization of the breast cancer immune microenvironment:  a feasibility study. A. Aukerman, M. Carri\`ere, C. Chen, K. Gardner, R. Rabad\'an, R. Vanguri.}
                          {\small\textit{SoCG}, \normalfont{2020}}
                          {}{}{\footnotesize Showed how to use persistence diagrams for the analysis of quantitative immunofluorescence images for breast cancer.}


	\subsection{Machine Learning and Statistics with Mapper}

		\cventry{-}{\normalfont\small Structure and Stability of the 1-Dimensional Mapper. M. Carri\`ere, S. Oudot}
                          {\textit{FoCM}, \normalfont{2017}}
                          {}{}{\footnotesize Defined appropriate metrics to prove the stability of the Mapper clustering algorithm.}

		\cventry{-}{\normalfont\small Local Equivalence and Intrinsic Metrics between Reeb graphs. M. Carri\`ere, S. Oudot}
                          {\small\textit{SoCG}, \normalfont{2017}}
                          {}{}{\footnotesize Established a local equivalence between the bottleneck and the Gromov Hausdorff distances for Reeb graphs.}		

		\cventry{-}{\normalfont\small Statistical Analysis and Parameter Selection for Mapper. M. Carri\`ere, B. Michel, S. Oudot}
                          {\small\textit{JMLR}, 2018}
                          {}{}{\footnotesize Defined confidence intervals and convergence rates for the Mapper clustering algorithm.}

		\cventry{-}{\normalfont\small Two-Tier Mapper: a user-friendly clustering method for global gene expression based on topology. R. Jeitziner, 
                           M. Carri\`ere, J. Rougemont, S. Oudot, K. Hess, C. Brisken}
                          {\small\textit{Bioinformatics}, \normalfont{2019}}
                          {}{}{\footnotesize Used a modified Mapper clustering algorithm for genomics.}

		\cventry{-}{\normalfont\small Topological Data Analysis of single-cell Hi-C contact maps. M. Carri\`ere, R. Rabad\'an}
                          {\small\textit{Abel Symposium}, \normalfont{2019}}
                          {}{}{\footnotesize Used the Mapper clustering algorithm to establish confidence regions for single-cell Hi-C contact maps.}

		\cventry{-}{\normalfont\small Approximation of Reeb spaces with Mappers and applications to stochastic filters. M. Carri\`ere, B. Michel}
                          {\small\textit{Preprint}, \normalfont{2019}}
                          {}{}{\footnotesize Studied applications of Mapper in machine learning.}	
		
	\subsection{Optimal transport}
	%\section{Submitted articles}

	
		\cventry{-}{\normalfont\small MREC: a fast and versatile framework for aligning and matching point clouds with applications to single cell molecular data. A. Blumberg, M. Carri\`ere, M. Mandell, R. Rabad\'an, S. Villar}
                          {\small\textit{Preprint}, \normalfont{2020}}
                          {}{}{\footnotesize Established recursive approximation scheme for fast computation of optimal transport.}	

		
                		
		
	\section{Skills}
		\cventry{Languages}{\normalfont French (native), English (professional TOEFL 627/677), Spanish (B1 level)}{}{}{}{}{}
		\cventry{Code}{\normalfont C++, Python (proficient), R, Matlab (prior experience)}{}{}{}{}{}

	\section{Coding projects}
		

		\cventry{-}{\normalfont Cover complex module of the C++/Python GUDHI library}{\url{http://gudhi.gforge.inria.fr/doc/latest/group__cover__complex.html}}{}{}{}
		\cventry{-}{\normalfont Representations module of the C++/Python GUDHI library}{\url{https://gudhi.inria.fr/python/3.1.0.rc1/representations.html}}{}{}{}
		\cventry{-}{\normalfont PersLay: a neural network layer for optimizing vectorizations of persistence diagrams}{\url{https://github.com/MathieuCarriere/perslay}}{}{}{}
		\cventry{-}{\normalfont MREC: a fast computational tool for optimal transport and applications to genomics}{\url{https://github.com/MathieuCarriere/mrec}}{}{}{}
		\cventry{-}{\normalfont My other projects can be found on my GitHub account}{\url{https://github.com/MathieuCarriere}}{}{}{}
	\section{Grants}
		%\cventry{}{\normalfont I received two 1000\euro{} grants, one from the DAAD exchange program, and one from the STIC doctoral school, for a 2-months visit at TU Munich, Germany,
                %        in the team of Ulrich Bauer}{}{}{}{}
		%\cventry{-}{\normalfont Funding Support from ATMCS 2016}{}{}{}{}
		\cventry{-}{\normalfont Mobility Grant (1000 euros) from the DAAD exchange program}{}{}{}{}
		\cventry{-}{\normalfont Mobility Grant (1000 euros) from the STIC doctoral school}{}{}{}{}
		\cventry{-}{\normalfont Best Scientific Contribution 2017 (2nd Prize -- 600 euros) from the STIC doctoral school}{}{}{}{}
 		\cventry{-}{\normalfont Funding Support (1800 dollars) from ICML 2017}{}{}{}{}
		\cventry{-}{\normalfont Thiess\'e de Rosemont / Schneider Prize (10,000 euros) from Chancellerie des Universit\'es de Paris}{}{}{}{}
		%\cventry{-}{\normalfont Funding Support from JGA 2017}{}{}{}{}

	\section{Teaching Activities}
		\cventry{}{\normalfont I was a teaching assistant for the following courses}{}{}{}{}
		\cventry{2015--2017}{\normalfont{\textit{Topological Data Analysis,}} \normalfont{Ecole Polytechnique, Palaiseau, France}}{}{}{}{}
		\cventry{2016--2017}{\normalfont{\textit{Basics of Algorithmic and Progamming,}} \normalfont{Ecole Polytechnique, Palaiseau, France}}{}{}{}{}

	\section{Presentations for Workshops and Conferences}
		\cventry{}{\normalfont I gave presentations at the following international conferences}{}{}{}{}
		%\cventry{06/2015}{\normalfont{\textit{Symposium on Geometry Processing,}} \normalfont{TU Graz, Graz, Austria}}{}{}{}{}
		\cventry{11/2015}{\normalfont{\textit{Journ\'ees de G\'eom\'etrie Algorithmique,}} \normalfont{IESC, Carg\`ese, Corsica}}{}{}{}{}
		\cventry{12/2015}{\normalfont{\textit{Computational and Methodological Statistics,}} \normalfont{London University, London, UK}}{}{}{}{}
		\cventry{04/2016}{\normalfont{\textit{Stochastic Geometry and its Applications,}} \normalfont{Universit\'e de Nantes, Nantes, France}}{}{}{}{}
		%\cventry{Jul. 2016}{\normalfont{\textit{Mathematical Methods for High Dimensional Data Analysis,}} \normalfont{TU Munich, Munich, Germany}}{}{}{}{}
		%\cventry{06/2016}{\normalfont{\textit{Symposium on Computational Geometry,}} \normalfont{Tufts University, Boston, USA}}{}{}{}{}
		\cventry{07/2016}{\normalfont{\textit{Applied Topology: Methods, Computation and Science,}} \normalfont{Politecnica di Torino, Torino, Italy}}{}{}{}{}
		%\cventry{08/2017}{\normalfont{\textit{International Conference on Machine Learning,}} \normalfont{Convention Centre, Sydney, Australia}}{}{}{}{}
		\cventry{09/2017}{\normalfont{\textit{France Japan Machine Learning Workshop,}} \normalfont{ENS Ulm, Paris, France}}{}{}{}{}
		\cventry{10/2017}{\normalfont{\textit{Amazon Graduate Research Symposium,}} \normalfont{Amazon Meeting Center, Seattle, USA}}{}{}{}{}
		\cventry{12/2017}{\normalfont{\textit{Journ\'ees de G\'eom\'etrie Algorithmique,}} \normalfont{Centre Paul Langevin, Aussois, France}}{}{}{}{}
		\cventry{08/2018}{\normalfont{\textit{TRIPODS Bootcamp on Topology and Machine Learning,}} \normalfont{ICERM, Providence, USA}}{}{}{}{}
		\cventry{01/2019}{\normalfont{\textit{AMS Special Session on Topological Data Analysis,}} \normalfont{Convention Center, Baltimore, USA}}{}{}{}{}
		\cventry{02/2019}{\normalfont{\textit{Applied Algebraic Topology Research Network,}} \normalfont{via Bluejeans}}{}{}{}{}
		\cventry{04/2019}{\normalfont{\textit{Topology, Geometry and Data Seminar,}} \normalfont{OSU, Columbus, USA}}{}{}{}{}
		\cventry{06/2019}{\normalfont{\textit{Symposium on Computational Geometry,}} \normalfont{PSU, Portland, USA}}{}{}{}{}
		%\cventry{06/2019}{\normalfont{\textit{Minisymposium on Computational Topology,}} \normalfont{PSU, Portland, USA}}{}{}{}{}
		\cventry{07/2019}{\normalfont{\textit{International Congress on Industrial and Applied Mathematics,}} \normalfont{UPV, Valencia, Spain}}{}{}{}{}
		\cventry{01/2020}{\normalfont{\textit{TDA Meeting,}} \normalfont{UF, Gainesville, USA}}{}{}{}{}
		\cventry{03/2020}{\normalfont{\textit{Probability and Society Symposium,}} \normalfont{Columbia University, New-York, USA}}{}{}{}{}
		\cventry{05/2020}{\normalfont{\textit{SIAM Conference on Mathematics of Data Science,}} \normalfont{Cincinnati, USA}}{}{}{}{}

	\section{References}

\cvtriplecolumn{\cvreference{Steve Oudot}
    {DataShape team}
    {Inria Saclay}
    {91120 Palaiseau, France}
    {}
    {steve.oudot@inria.fr}
    {+33 174 854 216}
}
{\cvreference{Marco Cuturi}
    {CREST - ENSAE}
    {Universit\'e Paris-Saclay}
    {91120 Palaiseau, France}
    {}
    {marco.cuturi@ensae.fr}
    {+33 170 266 857}
}
{\cvreference{Ra\'ul Rabad\'an}
    {Systems Biology Department}
    {Columbia University}
    {New-York, USA}
    {}
    {rr2579@columbia.edu}
    {}
}
    
\end{document}
\grid
